%UnlimitedFonts
	\def\hmmax{0}
	\def\bmmax{0}
%SVG
	\usepackage{svg}
%Tables
	\usepackage{array,booktabs,tabularx,multirow}
	\newcolumntype{C}[1]{>{\hsize=#1\hsize%
		\centering\arraybackslash}X}
%Math&Fonts
	\let\latexointop\ointop
	\usepackage{mathtools,amssymb,bm % basics
		,physics,siunitx,slashed % physics
		,esint,nicefrac,extarrows,mathrsfs % more symbols
		,calligra,romannum,dsfont,fourier-orns % nice fonts
		,eqnarray,resizegather,empheq % more envs
		,relsize,stackengine % utils
	}
%	\usepackage{amsthm}
	\usepackage[scr=esstix]{mathalfa}
	\usepackage[only,sslash]{stmaryrd}
	%DisplaySetup
	\newcommand*\bbox[1]{\fbox{\hspace{1em}\addstackgap[5pt]{#1}\hspace{1em}}}
	\empheqset{box=\bbox}
	\mathtoolsset{showonlyrefs}
%Utils
	%Legacy \oint
	\let\ointop\undefined
	\let\ointop\latexointop
	%Calligra
	\DeclareMathAlphabet{\mathcalligra}{T1}{calligra}{m}{n}
	\DeclareFontShape{T1}{calligra}{m}{n}{<->s*[2.2]callig15}{}
	%CosmeticTweaks
	\newcommand\inlineeqno{\stepcounter{equation}\ (\theequation)}
	\newcommand\scalemath[2]{\scalebox{#1}{\mbox{\ensuremath{\displaystyle #2}}}}
	\newcommand\raisemath[2]{\raisebox{#1\depth}{${#2}$}}
	\newfontfamily\signature{Vladimir Script}
	\newcommand{\newparagraph}{\pagebreak[3]
		\noindent\hfil%
		\raisebox{-4pt}[10pt][10pt]{\leafright~\qquad~\leafleft}%
		\par\nopagebreak%
	}
%CustomCmds
	%Brackets
	\DeclarePairedDelimiter\ave{\langle}{\rangle}
	\DeclarePairedDelimiterX\inprod[2]{\langle}{\rangle}{#1,#2}
	%Basics
	\newcommand{\mbb}[1]{\mathbb{#1}}
	\newcommand{\mrm}[1]{\mathrm{#1}}
	\newcommand{\mcal}[1]{\mathcal{#1}}
	\newcommand{\mscr}[1]{\mathscr{#1}}
	\newcommand{\tup}[1]{\textup{#1}}
	\newcommand{\mop}[1]{\operatorname{#1}}
	%Extras
	\newcommand{\scriptr}{\mathcalligra{r}\,}
	\newcommand{\rvector}{\pmb{\mathcalligra{r}}\,}
	\newcommand{\hodgedual}{\operatorname{\star}}
	\newcommand{\dual}{\ \xlongleftrightarrow{\ \textrm{dual}\ }\ }
	\newcommand{\idty}{\mathds{1}}
	\newcommand{\proj}[1]{\operatorname{%
		proj_{\mathit{#1}}}}
	\newcommand{\propsim}{\mathbin{\ensurestackMath{
		\stackunder[2pt]{\propto}{\sim}
	}}}
	\newcommand{\textbox}[1]{\fbox{#1}}
	\newcommand{\pdd}[1]{\operatorname{\partial_{\mathnormal{#1}}}}
	\newcommand{\cdd}{\operatorname{D}\!}
	\newcommand{\cdv}[1]{\operatorname{%
		\nabla_{\!\mathit{#1}\!}}}
	\newcommand{\ldv}[1]{\operatorname{%
		\mcal{L}_{\!\mathit{#1}\!}}}
	\newcommand{\ric}[1]{\operatorname{%
		Ric}\!\pqty{#1}}
%Hacks
	% physics.sty <texmf-dist/tex/latex/physics/>
	% USER: more spacing around Dirac's middle vert
	\newcommand{\xmiddle}[1]{\mspace{1mu}\middle#1\mspace{1mu}}
	\DeclareDocumentCommand\innerproduct{ s m g }
	{ % Inner product
		\IfBooleanTF{#1}
		{ % No resize
			\IfNoValueTF{#3}
			{\vphantom{#2}\left\langle\smash{#2}\xmiddle\vert\smash{#2}\right\rangle}
			{\vphantom{#2#3}\left\langle\smash{#2}\xmiddle\vert\smash{#3}\right\rangle}
		}
		{ % Auto resize
			\IfNoValueTF{#3}
			{\left\langle{#2}\xmiddle\vert{#2}\right\rangle}
			{\left\langle{#2}\xmiddle\vert{#3}\right\rangle}
		}
	}
